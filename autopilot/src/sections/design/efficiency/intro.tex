Though it is a single board, the Raspberry Pi operates more like a computer than a microcontroller. It most commonly runs a custom Linux distribution, Raspbian, allowing Autopilot to use Python across the whole system. Using an interpreted language like Python running on Linux has inherent performance drawbacks compared to compiled languages running on embedded microprocessors. In practice these drawbacks are less profound than they appear on paper: Python's overhead is negligible on modern processors\sidenote{and improvements to CPython in \href{https://docs.python.org/3.11/whatsnew/3.11.html\#faster-cpython}{Python 3.11} and onwards will bring overhead close to zero\citep{andersonGuidoVanRossum2021}}, jitter and performance can be improved by wrapping \hyperref[sec:lowlevel]{compiled code}, etc. While we view the gain in accessibility and extensibility of a widely used high-level language like Python as outweighing potential performance gains from using a compiled language, Autopilot is nevertheless designed to maximize computational efficiency.