%!TEX root=../../autopilot.tex
\newthought{Autopilot distributes experiments} across a network of Raspberry Pis,\sidenote{\href{https://www.raspberrypi.org/products/raspberry-pi-4-model-b/}{Raspberry Pi model 4B}, see \hyperref[hwtab]{Table \ref*{hwtab}}} a type of inexpensive single-board computer.

\vspace{12pt}

\textbf{Autopilot has three primary design principles:}

\begin{enumerate}
    \item \hyperref[sec:efficiency]{\textbf{Efficiency}} - Autopilot should minimize computational overhead and maximize use of hardware resources.
    \item \hyperref[sec:flexibility]{\textbf{Flexibility}} - Autopilot should be transparent in all its operations so that users can expand it to fit their existing or desired use-cases. Autopilot should provide clear points of modification and expansion to reduce local duplication of labor to compensate for its limitations.
    \item \hyperref[sec:reproducibility]{\textbf{Reproducibility}} - Autopilot should maximize system transparency and minimize the potential for the black-box of local reprogramming. Autopilot should maximize the information it stores about its operation as part of normal data collection.
\end{enumerate}
