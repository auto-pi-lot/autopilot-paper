%!TEX root=../../autopilot.tex



\newthought{We will likely never view} Autopilot as "finished." Autopilot---like all open-source software---is an evolving project, and this paper captures it in a snapshot of its development. We are invested in its development, and will be continually working to fix bugs, make its use more elegant, and add new features in collaboration with other researchers. 

We expect that as the codebase matures and other researchers use Autopilot in new, unexpected ways that some fundamental elements of its structure may evolve. We have built version logging into the structure of the system so that changes will not compromise the replicability of experiments (see \textbf{Versioning and Containerization} below). While there will inevitably be breaking changes, these will be transparently documented, \href{https://docs.auto-pi-lot.com/en/latest/changelog/index.html}{announced in release notes}, and indicated with \href{https://semver.org/}{semantic versioning} in order to alert users and describe how to adapt as needed. 

We recognize the risk and inertia of retooling lab infrastructure, and there is still much work to be done on Autopilot. We welcome any brave early adopters with a taste for the challenge of filling the collective need for a generalizable experimental framework. Trying it is ultimately as risky as buying a Raspberry Pi. 

\vspace{12pt}

\noindent The current list of major changes (loosely summarizing the \href{https://docs.auto-pi-lot.com/en/latest/todo.html}{todo} page in the docs):


\begin{enumerate}[ref=\thechapter.\arabic*]
    \item \label{future:rust}\textbf{Python, Meet Rust} - rewriting mature things in rust
    \item \label{item:othertools} \textbf{Integration with Other Software} - We will make Autopilot capable of natively recording electrophysiological data by integrating with Open Ephys\citep{siegleOpenEphysOpensource2017}. We also are interested in tightly integrating other recent tools like DeepLabCut\citep{nathUsingDeepLabCut3D2019} and MoSeq\citep{wiltschkoMappingSubSecondStructure2015} to make Autopilot a unified platform for complex and naturalistic behavioral experiments.
    \item \label{future:metastructure}\textbf{Metastructure \& Code Quality} - unified object system, cleaning up as we go!
    \item \textbf{Data} - NWB!\citep{rubelNWBAccessibleData2019} Datajoint! releasing v0.5.0 as an alpha version because this isn't quite done!
    \item \textbf{Provenance} - 
    \item \textbf{Tasks} - We look forward to collaborating with other researchers to expand the available library of tasks. While the two-alternative forced choice and go/no-go tasks we have implemented are common, we designed Autopilot to be capable of performing \textit{any} behavioral experiment.
    \item \label{future:network}\textbf{Mesh Networking} - The tree structure of Autopilot's networking was built to enforce simplicity of its messaging protocol, but it limits the ability for data to be shared efficiently between a large number of pilots because communication has to be routed through a hub terminal. We will implement a true mesh network architecture by implementing a distributed hash table, allowing agents to directly communicate with one another without explicit configuration. We also will implement a peer-to-peer data protocol akin to Bittorrent to allow efficient distribution of data across a swarm of agents.
    \item \textbf{Tests} - At release, Autopilot has no unit tests. To make the codebase easier to maintain, we aim to reach 100\% coverage by the first stable release of the program (v1.0).
    \item \label{future:gui} \textbf{Save the GUI Plz}
    \item \label{future:plugins} \textbf{Plugins} - we want to be permissive, but we still need a way of specifyinng dependencies and all that. Also need to still complete all the actual management stuff like getting plugins and etc.
    \item \label{future:wiki} \textbf{Wiki} - generate data models to upload to wiki, 
\end{enumerate}
\end{fullwidth}

