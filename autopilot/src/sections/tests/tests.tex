
\newthought{We have been testing and refining Autopilot} since we built our swarm of 10 training boxes in the spring of 2019. In that time 178 mice\sidenote{All procedures were performed in accordance with National Institutes of Health guidelines, as approved by the University of Oregon Institutional Animal Care and Use Committee.} have performed over 6 million trials on a range of tasks. While Autopilot is still relatively new, it is by no means untested.

In this section we will present a set of basic performance benchmarks while also showing several of the different ways that Autopilot can be used. The code for all of the following tests is available as a \href{https://github.com/auto-pi-lot/plugin-tests}{plugin} that is further documented on the \href{https://wiki.auto-pi-lot.com/index.php/Plugin:Autopilot_Paper}{wiki}, and runs on a prerelease of v0.5.0 (specific commit hashes will be referenced where relevant). 

\begin{margintable}[-0.5cm]
\caption{Latency Test Materials}
\label{tab:materials}
\noindent\begin{tabularx}{\linewidth}{lX}%
\toprule
\textbf{Hardware} & \href{https://www.raspberrypi.org/products/raspberry-pi-4-model-b/}{Raspberry Pi 4}\\
Soundcard & \href{https://www.hifiberry.com/shop/boards/hifiberry-amp2/}{Hifiberry Amp2} \\
IR Break Sensor & \href{https://www.digikey.com/product-detail/en/tt-electronics-optek-technology/OPB901L55/365-1767-ND/1637490}{TT Electronics OPB901L55}\\
Speaker & \href{https://www.parts-express.com/hivi-rt13we-isodynamic-tweeter--297-421}{HiVi RT1.3WE}\\
Oscilloscope & \href{https://download.tek.com/manual/071181702web.pdf}{Tektronix TDS 2004B}\\
Router & \\
\midrule
Python & version \\
\midrule
\textbf{Software} & \\
RaspiOS version & \\
\midrule
\textbf{Analysis} & \\
R & (version) \\
ggplot2 & (version) \\
dplyr & (version) \\
\bottomrule
\end{tabularx}
\end{margintable}
