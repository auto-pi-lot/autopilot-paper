
\newthought{We have been testing and refining Autopilot} since we built our swarm of 10 training boxes in the spring of 2019. In that time 178 mice\sidenote{All procedures were performed in accordance with National Institutes of Health guidelines, as approved by the University of Oregon Institutional Animal Care and Use Committee.} have performed over 6 million trials on a range of tasks. While Autopilot is still relatively new, it is by no means untested.

In this section we will present a set of basic performance benchmarks while also showing several of the different ways that Autopilot can be used. The code for all of the following tests is available as a \href{https://github.com/auto-pi-lot/plugin-tests}{plugin} that is further documented on the \href{https://wiki.auto-pi-lot.com/index.php/Plugin:Autopilot_Paper}{wiki}, and runs on a prerelease of v0.5.0. 

\begin{margintable}[-0.5cm]
\caption{General Materials}
\label{tab:materials}
\noindent\begin{tabularx}{\linewidth}{lX}%
\toprule
\textbf{Hardware} & \\
Raspi & \href{https://www.raspberrypi.org/products/raspberry-pi-4-model-b/}{Raspberry Pi 4b}\\
Oscilloscope & \href{https://wiki.auto-pi-lot.com/index.php/Rigol\_DS1054Z}{Rigol DS1054Z} \\
\midrule
\textbf{Software} & \\
Autopilot & link to v0.5.0a \\
Plugin & \href{https://wiki.auto-pi-lot.com/index.php/Plugin:Autopilot\_Paper}{Autopilot\_Paper} \\
Python & 3.9.12 \\
RaspiOS & Bullseye \href{https://downloads.raspberrypi.org/raspios_lite_armhf/images/raspios\_lite\_armhf-2022-04-07/}{22-04-04} (lite) \\
Lockfile & Link to v0.5.0a lockfile \\
\midrule
\textbf{Analysis} & \\
R & (version) \\
ggplot2 & (version) \\
dplyr & (version) \\
pandas & (version) \\
numpy & (version) \\ 
\bottomrule
\end{tabularx}
\end{margintable}
